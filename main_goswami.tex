%%% -----------------------------------
%%% -----------------------------------
%%% ----- META DATA (TITLE & AUTHORS)
%%% -----------------------------------
%%% -----------------------------------
\include{preamble}
\usepackage{lipsum}
\title{Title of the project}
\author{Abc Xyz and Def Uvw}
\date{\today{}}
\begin{document}
\maketitle

%%% -----------------------------------
%%% -----------------------------------
%%% ----- ABSTRACT
%%% -----------------------------------
%%% -----------------------------------

\begin{abstract}
Brief summary of the main idea, its motivation, the main results, and outlook. Should not be more than 150 words. \lipsum[2-2]
\end{abstract}

%%% -----------------------------------
%%% -----------------------------------
%%% ----- MOTIVATION AND BACKGROUND
%%% -----------------------------------
%%% -----------------------------------

\section{Motivation and Background}
\label{sec:motivation_background}

\lipsum[2-2]

Climate networks can be constructed using different `similarity'
measures. Let \(\vec{x}_i = (x_{i1}, x_{i2}, \dots, x_{iN_t})^T\) be a
time series of a climatic observable such as surface air temperature or
sea level pressure measured at a location
\(i; i \in \{1, 2, \dots, N_x\}\). If we assume that each \(\vec{x}_i\)
is normalized to mean zero and standard deviation one, the Pearson's
correlation matrix can simply be written as,

\begin{eqnarray}
\mathbf{C}^{PC} = \mathbf{X^T} \mathbf{X},
\label{eq:corrmat}
\end{eqnarray}
where
\(\mathbf{X} = [ \vec{x}_1, \vec{x}_2, \dots, \vec{x}_{N_x} ]\). A
`climate network' is constructed by thresholding \(\mathbf{C}\) to
remove low correlation values which are likely due to noise. The
adjacency matrix of the climate network is, 
\begin{eqnarray}
\mathbf{A} = F(\mathbf{C}^{PC}) - \mathbb{I}
\label{eq:adjmat}
\end{eqnarray}
where \(F\) is an element-wise operator on the correlation matrix
\(\mathbf{C}^{PC}\) which is typically a Heaviside function, 
\begin{eqnarray}
F(\mathbf{C}^{PC}) = 
\begin{cases}
1, \quad |c^{PC}_{ij} - \epsilon | > 0\\
0, \quad \textrm{otherwise}
\end{cases}
\label{eq:heaviside}
\end{eqnarray}
where \(c_{ij}^{PC} \in \mathbf{C}^{PC}\) and \(\epsilon\) is a
threshold above which correlation values are less likely to be caused by
random noise.

\lipsum[4-4]

We also want to cite this paper~\cite{goswami2018abrupt} at this point. We refer the reader to~\cite{ghil2002advanced} and \cite{reichstein2019deep} for an overview.

%%% -----------------------------------
%%% -----------------------------------
%%% ----- DATA AND METHODS
%%% -----------------------------------
%%% -----------------------------------

\section{Data and Methods}
\label{data_methods}

\lipsum[2-4]

%%% -----------------------------------
%%% -----------------------------------
%%% ----- RESULTS AND DISCUSSION
%%% -----------------------------------
%%% -----------------------------------

\section{Results and discussion}
\label{results_discussion}

\lipsum[2-3]
\begin{figure}[!tbh]
    \centering
    \includegraphics[width=1.15\textwidth]{201_firstlook_stationdata_timecoveragedetail.pdf}
    \caption{\textbf{Station Coverage}. Data coverage of the weather stations around Heshang Cave between 1940 to 2020.}
    \label{fig:station_coverage}
\end{figure}

\lipsum[5-5]

\begin{figure}[!tbh]
    \centering
    \includegraphics[width=1.15\textwidth]{301_effp_dr_linearmodel_yearly_fits_lognorm.pdf}
    \caption{Yearly distribution of drip rates.}
    \label{fig:my_label}
\end{figure}

\lipsum[9-12]

%%% -----------------------------------
%%% -----------------------------------
%%% ----- CONCLUSION AND OUTLOOK
%%% -----------------------------------
%%% -----------------------------------

\section{Conclusion and outlook}
\label{concl_outlook}

\lipsum[1-2]

%%% -----------------------------------
%%% -----------------------------------
%%% ----- REFERENCES
%%% -----------------------------------
%%% -----------------------------------

\bibliography{refs}
\bibliographystyle{ieeetr}


\end{document}